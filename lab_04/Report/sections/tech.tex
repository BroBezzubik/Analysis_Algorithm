\documentclass[../main.tex]{subfiles}
\begin{document}
	
	В данном разделе указаны минимальные системные требования. Описан используемый язык и среда разработки.
	Приведен листинг алгоритмов умножения матриц

\subsection{Минимальные требования}

	Минимальные системные требования: PC с операционной системой Windows XP/Vista/7/8/10. Требуются устройства ввода: клавиатура, мышь. Устройство вывода: монитор.

\subsection{Выбор языка и среды разработки}

	Для решения данной поставленной задачи, мной был выбран язык Python \href{https://docs.python.org/3/whatsnew/3.8.html}{3.8} по удобности и знания. Так же использовалась среда \href{https://www.jetbrains.com/pycharm/}{PyCharm 2019}
	
\subsection{Листинг}

	\lstinputlisting[firstline=3, lastline=20, 
					caption=Классический алгоритм умножения, 
					label=listing:1]{../../trash/classic.py}
	
	\lstinputlisting[firstline=6, lastline=31, 
					caption=Алгоритм Винограда с поддержкой параллельности, 
					label=listing:2]{../../test.py}
					
	\lstinputlisting[firstline=33, lastline=54, 
					caption= Функция распараллеливания вычислений алгоритма Винограда по столбцам, 
					label=listing:2]{../../test.py}
					
	
\end{document}