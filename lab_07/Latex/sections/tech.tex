\documentclass[../main.tex]{subfiles}

\begin{document}
	
\subsection{Требования к программному обеспечению}

	Минимальные системные требования: PC с операционной системой Windows версии XP/Vista/7/8/10. Требуются устройства ввода: клавиатура, мышь. 

\subsection{Средства реализации}

	Для выполнения работы был выбран язык программирования Python ввиду его простоты. И SublimeText3 \href{https://www.sublimetext.com/}{text}

\subsection{Интерфейс}

	Интерфейс из себя представляет простую консоль, где пользователю ничего делать ненужно.
	
\subsection{Листинг}
	
	В данном подразделе приведены листинги программ.
	\begin{enumerate}
		\item алгоритм Кнута - Мориса - Прата \ref{list:1}
		\item алгоритм Бойера - Мура \ref{list:2}
	\end{enumerate}

	\begin{lstlisting}[caption=Алгоритм Кнута-Мориса-Прата, label=list:1]
	
		def prefix(s):
		    v = [0] * len(s)
		    for i in range(1, len(s)):
		        k = v[i - 1]
		        while k > 0 and s[k] != s[i]:
		            k = v[k - 1]
		        if s[k] == s[i]:
		            k = k + 1
		        v[i] = k
		    return v

		def kmp(s, t):
		    index = -1
		    f = prefix(s)
		    k = 0
		    for i in range(len(t)):
		        while k > 0 and s[k] != t[i]:
		            k = f[k - 1]
		        if s[k] == t[i]:
		            k = k + 1
		        if k == len(s):
		            index = i - len(s) + 1
		            break
		    return index
	\end{lstlisting}
	
	\begin{lstlisting}[label=list:2, caption=Алгоритма Бойера-Мура]
		def badCharHeuristic(string, size):
    		badChar = [-1] * 256
	    	
	    	for i in range(size):
	        	badChar[ord(string[i])] = i

    		return badChar


		def search(txt, pat):
		    m = len(pat)
		    n = len(txt)
		    badChar = badCharHeuristic(pat, m)
		    s = 0
		    while s <= n - m:
		        j = m - 1

		        while j >= 0 and pat[j] == txt[s + j]:
		            j -= 1

		        if j < 0:
		            return s  # Return only first entry

		        else:
		            s += max(1, j - badChar[ord(txt[s + j])])

		    return -1
	\end{lstlisting}
	
	В данном разделе были приведены листинги алгоритмов Кнута-Мориса-Прата и Бойера-Мура на языке программирования Python, а также была приведена функция тестирования этих алгоритмов.

	
\end{document}