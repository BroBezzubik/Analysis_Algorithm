\documentclass[../main.tex]{subfiles}

\begin{document}
	
	Задача коммивояжёра - одна из самых известных задач транспортной логистики. Одно из первых её решений было предложено У.Гамильтоном в XIX веке.
	Суть задачи заключается в следующем:\\
	необходимо найти оптимальный (кратчайший) путь через некоторые пункты, проходя каждый по одному разу. Мерой выгодности маршрута могут быть минимальное время, минимальные расходы на дорогу или минимальная длина пути. \cite{TSP} \\
	
	В последние годы интенсивно развивается научное направление, называемое природными вычислениями, объединяющее математические методы, в основе которых заложены природные механизмы и принципы принятия решений, сформированные во флоре и фауне на протяжении миллионов лет. 
	Имитация муравьиной колонии составляет основу муравьиных алгоритмов. 
	Колония муравьёв рассматривается как многоагентная система, в которой каждый муравей рассматривается как отдельный агент, функционирующий автономно по простым правилам. \cite{AntAlg}
	
\end{document}