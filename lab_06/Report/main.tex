\documentclass[a4paper, 14pt]{article}
\usepackage[utf8]{inputenc}
\usepackage[russian]{babel}
\usepackage{graphicx}
\usepackage{float}
\usepackage{amssymb}
\usepackage{amsmath}
\usepackage{pgfplots}
\usepackage{hyperref}
\usepackage{subfiles}
\usepackage[shortlabels]{enumitem}
\usepackage{titlesec}
\usepackage{algorithm}
\usepackage{algpseudocode}


\titleformat*{\section}{\LARGE\bfseries}
\titleformat*{\subsection}{\Large\bfseries}
\titleformat*{\subsubsection}{\large\bfseries}
\titleformat*{\paragraph}{\large\bfseries}
\titleformat*{\subparagraph}{\large\bfseries}


\usepackage{biblatex}[
backend = biber,
style=alphabetic,
sorting=ynt]
\addbibresource{./library/document}

% Настройка Листинга
\usepackage{listings}
\lstset{
	language=Python,
	numbers=left,
	frame=single,
	breaklines=true,
	breakatwhitespace=true,
	title=lstname,
	tabsize=2,
	framexrightmargin=20mm
}


\DeclareUnicodeCharacter{03BB}{\ensuremath{\lambda}}
\DeclareUnicodeCharacter{03B7}{\ensuremath{\eta}}
\DeclareUnicodeCharacter{03C4}{\ensuremath{\tau}}
\DeclareUnicodeCharacter{03C1}{\ensuremath{\rho}}


\begin{document}
	\begin{titlepage}
		\begin{center}
			\begin{LARGE}
				Отчет по лабораторной работе №6\\
				по курсу "Анализ алгоритмов"\\
				по теме "Муравьиный алгоритм"
			\end{LARGE}
			
			\begin{Large}
				\vspace{10cm}
				Студент: Барсуков Н.М. ИУ7-56\\
				Преподаватель: Волкова Л.Л.,
				Строганов Ю.В.
			\end{Large}
		\end{center}
	\end{titlepage}
	
	\newpage
	\tableofcontents
	
	\newpage
	\section*{Введение}
	\subfile{sections/intro}
	
	\newpage
	\section{Аналитический раздел}
	\subfile{sections/analit}
	
	\newpage
	\section{Конструкторский раздел}
	\subfile{sections/constructor}
	
	\newpage
	\section{Технологический раздел}
	\subfile{sections/tech}
	
	\newpage
	\section{Иследовательский раздел}
	\subfile{sections/diss}
	
	\newpage
	\section{Заключение}
	\subfile{sections/end}
	
	\newpage
	\addcontentsline{toc}{section}{Список использованных источников}
	\begin{thebibliography}{00} % Список литературы
		\bibitem{TSP}
		Задача коммивояжера - метод ветвей и границ. -- URL: $http://galyautdinov.ru/post/zadacha-kommivoyazhera$
		\bibitem{AntAlg}
		Штовба С. Д. Муравьиные алгоритмы, Exponenta Pro. Математика в приложениях. 2004. № 4
		\bibitem{ACS}
		M. Dorigo \& L. M. Gambardella, 1997. «Ant Colony System: A Cooperative Learning Approach to the Traveling Salesman Problem». IEEE Transactions on Evolutionary Computation, 1 (1): 53-66.
		\bibitem{RECA}
		М.В.Ульянов. Ресурсно-эффективные компьютерные алгоритмы $//$ 2007. - Раздел III. - Глава 7. - С.195-206.
	\end{thebibliography}
\end{document}