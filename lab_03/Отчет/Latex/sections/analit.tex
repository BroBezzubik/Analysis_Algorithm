В данном разделе описана поставленная цель и задачи необходимые для ее решения. Приведено описание трех алгоритмов сортировки. 

\subsection{Постановка задачи}
	
	Цель: Изучить и реализовать 3 версии алгоритма сортировки по выбору. Для одной из них требуется расписать нахождение зависимости количества	операций от размера массива и его изначального наполнения. Для двух других требуется лишь написать зависимость времени их выполнения от количества сортируемых элементов. Всего рассматривается 3 случая
	
	Для решения поставленной цели, необходимо выполнить следующие поставленные задачи:
	
	\begin{enumerate}[1)]
		\item изучить:
		\begin{enumerate}
			\item алгоритм сортировки выбором;
			\item алгоритм сортировки пузырьком (с флагом);
			\item алгоритм сортировки вставками.
		\end{enumerate}
		\item реализовать выше перечисленные алгоритмы;
		\item ввести модель оценки сложности алгоритма;
		\item провести анализ сложности;
		\item произвести замеры времени работы алгоритмов;
		\item провести анализ;
		\item сделать выводы
	\end{enumerate}

	Сортировка вставками \cite{McConnel} - сортировка, основная идея, который, заключается в том, что при добавлении нового элемента в уже отсортированный список, его стоит сразу вставлять в нужное место вместо того, чтобы вставлять его в произвольное место, а затем снова сортировать весь список. 
	Сортировка вставками считает первый элемент любого списка отсортированным списком длины 1.
	Двухэлементный отсортированный список создается добавлением второго элемента исходного списка в нужное место одно элементного списка содержащего первый элемент. 
	Теперь можно вставить третий элемент исходного списка в отсортированный двухэлементный список. 
	Этот процесс повторяется до тех пор, пока все элементы исходного списка не окажутся в расширяющейся отсортированной части списка. \\
	
	
	Сортировка пузырьком \cite{McConnel} - сортировка основной идеей, которой является выталкивание маленьких значений на вершину списка в то время, как большие значения опускаются вниз. 
	У пузырьковой сортировки есть много различных вариантов. \\
	
	
	Алгоритм пузырьковой сортировки совершает несколько проходов по списку. 
	При каждом проходе происходит сравнение соседних элементов. 
	Если порядок соседних элементов неправильный, то они меняются местами. 
	Каждый проход начинается с начала списка. 
	Сперва сравниваются первый и второй элементы, затем второй и третий, потом третий и четвертый и так далее; элементы с неправильным порядком в паре переставляются. 
	При обнаружении на первом проходе наибольшего элемента списка он будет переставляться со всеми последующими элементами пока не дойдет до конца списка. 
	Поэтому при втором проходе нет необходимости производить сравнение с последним элементом.
	При втором проходе второй по величине элемент списка опустится во вторую позицию с конца. 
	При продолжении процесса на каждом проходе по крайней мере одно из следующих по величине значений встает на свое место. 
	При этом меньшие значения тоже собираются наверху.
	Если при каком-то проходе не произошло ни одной перестановки элементов, то все они стоят в нужном порядке, и исполнение алгоритма можно прекратить. 
	Стоит заметить, что при каждом проходе ближе к своему месту продвигается сразу несколько элементов, хотя гарантированно занимает окончательное положение лишь один. \\
	

	Сортировка выбором - алгоритм суть которого заключается в том что:
	\begin{enumerate}[1)]
		\item разделяем массив на 2 части: отсортированную и неотсортированную. На начальный момент времени отсортированная часть пуста;
		\item производим поиск минимального элемента в неотсортированной части массива и вставляем его в конец отсортированной части;
		\item повторяем второй шаг, пока размер не отсортированнойчасти массива не станет равным 1.
	\end{enumerate}

\subsection{Вывод}
В данном разделе была поставлена цель и описаны задачи необходимые для ее решения. Приведено описание алгоритмов