\documentclass[../main.tex]{subfiles}

\begin{document}

	В результате выполнения лабораторной работы была изучены задача коммивояжёра и способы её решения путём использования алгоритмов природных вычислений, получены навыки реализации муравьиного алгоритма, проведена его параметризация. 
	Аналитическая параметризация, приведённая в конструкторском разделе, была подкреплена экспериментами. Было выявлено, что наилучший результат муравьиный алгоритм получает при $\alpha = 0.5, \beta = 0.5, ρ = 0.5$. 
	Для коэффициента испарения феромона критичным оказалось даже изменение на 10\%. Параметры $\alpha$ и $\beta$ можно изменить до 10\% от этих значений при преобладании коэффициента $\beta$ (то есть 0.4 и 0.6 соответственно) без потери результата. 
	В случае достижения параметрами граничных значений муравьиный алгоритм вырождается. Если $\alpha = 0$, то алгоритм становится жадным, а в противоположном случае перестаёт зависеть от расположения городов и начинает работать случайным образом. 
	Если $ρ = 0$, то всё сведётся к жадному алгоритму, а если $ρ = 1$, то теряется зависимость от опыта предыдущих поколений колонии.

\end{document}