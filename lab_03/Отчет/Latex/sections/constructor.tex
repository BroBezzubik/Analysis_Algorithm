В данном разделе представлены схемы трех алгоритмов сортировки: сортировка пузырьком с флагом, сортировка вставками, сортировка выбором. Располагается описание сложности алгоритмов (взята из литературы). Сложность для сортировки вставками описана по выбранной модели в данном разделе.

\subsection{Алгоритмы}

	В данном подразделе представлены схемы трех алгоритмов сортировки:
	\begin{enumerate}[1)]
		\item сортировка пузырьком
		\item сортировка вставками
		\item сортировка выборкой
	\end{enumerate}

	\subfile{shema}

\subsection{Модель трудоемкости}
	
	Основные правила оценки трудоемкости:
	\begin{enumerate}
		\item Операции единичной трудоемкости: +, -, *, /, %, <, <=, =>, >, ==, ! =, [], + =, − =, ∗ =, = =, ++, −−;
		\item C - подобная модель оценки трудоемкости циклов:
		\begin{enumerate}
			\item 
			\lstinline|for (int i = 0; i < N; i++){}| \\
			$ A = 2 + N * (2 + A()) $;
			\item
			\lstinline|for (int i = 0; i < N + 1; i++){}| \\
			$ A = 3 + N * (3 + A()) $.
			
		\end{enumerate}
	\end{enumerate}

\subsection{Оценка сложности алгоритмов}
	
	\begin{enumerate}[1)]
		\item соритровка вставками:
		\begin{enumerate}
			\item Сравнений: 3Size;
			\item Сложений: 4size;
			\item Умножений Size;
			\item Присвоений: 3Size + 1;
			\item Обращений по индексу: 3size
			\item Memmove: Size.
		\end{enumerate}
		\item пузырек:
		\begin{enumerate}
			\item лучший случай: $O(M)$;
			\item худший случай: $O(M^2$).
		\end{enumerate}
		
		\item сортировка выбором:
		
		\begin{enumerate}
			\item лучший случай: $O(M)$;
			\item худший случай: $O(M^2$).
		\end{enumerate}
	\end{enumerate}
	
	