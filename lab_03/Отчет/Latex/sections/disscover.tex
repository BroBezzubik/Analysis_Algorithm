В данном разделе указаны характеристики машины на которой проводилось тестирование. Представлены замеры и результаты сравнения алгоритмов

\subsection{Характеристики оборудования}
	\begin{enumerate}
		\item Компьютер:
		\begin{enumerate}
			\item Тип компьютера   Компьютер с ACPI на базе x64;
			\item Операционная система   Microsoft Windows 10 Pro.
		\end{enumerate}
		\item Системная плата:
		\begin{enumerate}
			\item тип ЦП   DualCore Intel Core i5-6200U, 2700 MHz (27 x 100);
			\item системная плата   HP 8079;
			\item чипсет системной платы   Intel Sunrise Point-LP, Intel Skylake-U;
			\item системная память   8072 МБ (DDR4 SDRAM).
		\end{enumerate}
	\end{enumerate}

\subsection{Замеры и сравнение}
	
	Параметры замеров:
	\begin{enumerate}[1)]
		\item время вычисляется суммарно на 100 тестов;
		\item замеры производятся для:
		\begin{enumerate}
			\item идеальный случай;
			\item худший случай;
			\item произвольный.
		\end{enumerate}
		\item замеры сделаны для массивом размером от 1000 до 10000.
	\end{enumerate}

	Ниже представлены замеры трех алгоритмов сортировки: сортировка пузырьком\ref{tab:1}, сортировка вставками\ref{tab:2}, сортировка выбором\ref{tab:3}.
	
	\subfile{calculation}
	
	\subfile{graphicks}

	\begin{enumerate}
		\item лучший случай: Как мы можем заметить соритировка вставками и пузырьком, почти идеально работают для лучших случаев 0 или 31 один. Это в 416 эффективнее чем сортировка выбором.
		\item худший случай: Самый худший результат показала сортировка пузырьком: она в 5 раз хуже сортировки выбором и сортировки вставками
		\item произвольный случай: Самым худший результат показал сортировка пузырьком, она в 4.5 хуже сортировки Выбором и хуже в 8.5 раз сортировки вставками.
		
	\end{enumerate}
	
\subsection{Вывод}

	В данном разделе приведены результаты замеров в таблице и приведены графики. Произведено сравнение алгоритвом для разных вариантов данных.
	
	
	
	  
	