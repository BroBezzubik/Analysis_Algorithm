\documentclass[../main.tex]{subfiles}

\begin{document}
	
	В данном разделе приведены основные требования к конфигурации машины. 
	Описан используемый язык и среда разработки.
	Показан интерфейс.
	Приведен листинг программ.
	
\subsection{Минимальные требования}

	Минимальные системные требования: PC с операционной системой Windows XP/Vista/7/8/10. 
	Требуются устройства ввода: клавиатура, мышь. 
	Устройство вывода: монитор.

\subsection{Выбор языка и среды разработки}

	Для выполнения поставленной задачи выл выбран язык разработки \href{https://www.erlang.org/downloads}{Erlang/OTP 22.2} с целью изучения новых подходов в программировании.
	Так же была выбрана \href{https://www.jetbrains.com/idea/}{Intellij IDEA Education Edition} по причине доступности и удобства работы с выбранным языком.

\subsection{Интерфейс}

	Интерфейс программы представляет из себя простую консольную программу которая. Которая просит от пользователя:
	\begin{enumerate}
		\item начальное количество слов;
		\item конечное количество слов;
		\item шаг изменения
		\item длинна слова;
		\item количество прогонов
	\end{enumerate}

\subsection{Листинг}
	
	В данном подразделе приведены листинги реализаций двух подходов в выполнении программ: классический последовательный и конвейерный
	
	\lstinputlisting[caption = Листинг программы в классическом подходе]{../../programm/simpleway.erl}
	\lstinputlisting[caption = Листинг программы в конвейерном подходе]{../../programm/conveerway.erl}
	
\end{document}