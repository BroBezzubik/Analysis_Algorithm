\documentclass[a4paper, 14pt]{article}
\usepackage[utf8]{inputenc}
\usepackage[russian]{babel}
\usepackage{graphicx}
\usepackage{float}
\usepackage{amsmath}
\usepackage{pgfplots}
\usepackage{hyperref}
\usepackage{subfiles}
\usepackage[shortlabels]{enumitem}
\usepackage{titlesec}
\titleformat*{\section}{\LARGE\bfseries}
\titleformat*{\subsection}{\Large\bfseries}
\titleformat*{\subsubsection}{\large\bfseries}
\titleformat*{\paragraph}{\large\bfseries}
\titleformat*{\subparagraph}{\large\bfseries}

\usepackage{biblatex}[
backend = biber,
style=alphabetic,
sorting=ynt]
\addbibresource{./library/document}

\usepackage{listings}
\lstset{
	language=Erlang,
	numbers=left,
	frame=single,
	breaklines=true,
	breakatwhitespace=true,
	title=lstname,
	tabsize=2	
}


\begin{document}
	\begin{titlepage}
		\begin{center}
			\begin{LARGE}
				Отчет по лабораторной работе №5\\
				по курсу "Анализ алгоритмов"\\
				по теме "Конвейерный подход"
			\end{LARGE}
			
			\begin{Large}
				\vspace{10cm}
				Студент: Барсуков Н.М. ИУ7-56\\
				Преподаватель: Волкова Л.Л.,
				Строганов Ю.В.
			\end{Large}
		\end{center}
	\end{titlepage}
	
	\newpage
	\tableofcontents
	
	\newpage
	\section*{Введение}
	\subfile{sections/intro}
	
	\newpage
	\section{Аналитический раздел}
	\subfile{sections/analit}
	
	\newpage
	\section{Конструкторский раздел}
	\subfile{section/constructor}
	
	\newpage
	\section{Технологический раздел}
	\subfile{section/tech}
	
	\newpage
	\section{Иследовательский раздел}
	\subfile{section/diss}
	
	\newpage
	\section{Заключение}
	\subfile{section/end}
	
	\newpage
	\addcontentsline{toc}{section}{Список литературы}
	\begin{thebibliography}{4}
		
		\bibitem{McConnel}
		Дж. Макконнелл. Анализ алгоритмов. Активный обучающий подход.-
		М.:Техносфера, 2009.
		
		\bibitem{Erlang}
		Erlang/OTP 22.2 Электронный ресурс / Режим доступа https://erlang.org/doc/index.html. последнее обращение 25.12.2019
	
		\bibitem{Conveyer}
		Пензенский Государственный Университет / Конвейеризация и параллелизм. Конвейерная организация обработки данных. Простейшая организация конвейера и оценка его производительности. / Электронный ресурс / Режис доступа https://studfile.net/preview/3991398/page:17.
		Последнее обращение 25.12.2019
		
	\end{thebibliography}
\end{document}