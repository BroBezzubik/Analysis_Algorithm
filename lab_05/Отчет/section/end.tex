\documentclass[../main.tex]{subfiles}

\begin{document}
	
	В данной лабораторной работе были реализованны два различных подхода в выполнении задач: конвейерный и последовательный.
	Проведены замеры результатов эффективности методов которые показали, что не смотря на простоту выполняемой задачи преобразования строк, конвейерный метод оказывается эффективнее последовательного, хотя на малых количествах строк они и идут вровень. На пример на количестве строк равном 2000 классический подход работает в 1,033 медленнее конвейерного (на 0.562 секунды медленнее). Но чем больше слов тем больше разница, так для 10000 слов классический работает в 1,150 медленнее чем его коллега. (на 13 секунд медленнее).
	
\end{document}