\documentclass[../main.tex]{subfiles}

\begin{document}
	
	Конвейеризация или параллелизм? Вот в чем вопрос. Разработчики архитектуры компьютеров издавна прибегали к методам проектирования, известным под общим названием "совмещение операций", при котором аппаратура компьютера в любой момент времени выполняет одновременно более одной базовой операции. Этот общий метод включает два понятия: параллелизм и конвейеризацию. Хотя у них много общего и их зачастую трудно различать на практике, эти термины отражают два совершенно различных подхода. При параллелизме совмещение операций достигается путем воспроизведения в нескольких копиях аппаратной структуры. Высокая производительность достигается за счет одновременной работы всех элементов структур, осуществляющих решение различных частей задачи. \\
	
	Конвейеризация (или конвейерная обработка) в общем случае основана на разделении подлежащей исполнению функции на более мелкие части, называемые ступенями, и выделении для каждой из них отдельного блока аппаратуры. Так обработку любой машинной команды можно разделить на несколько этапов (несколько ступеней), организовав передачу данных от одного этапа к следующему. При этом конвейерную обработку можно использовать для совмещения этапов выполнения разных команд. Производительность при этом возрастает благодаря тому, что одновременно на различных ступенях конвейера выполняются несколько команд. Конвейерная обработка такого рода широко применяется во всех современных быстродействующих процессорах. \\
	
\end{document}