\documentclass[../main.tex]{subfiles}
\begin{document}
	
	Умножение матриц — это один из базовых алгоритмов, который широко применяется в различных численных методах, и в частности в алгоритмах машинного обучения. 
	Многие реализации прямого и обратного распространения сигнала в сверточных слоях нейронной сети базируются на этой операции. 
	Так порой до 90-95\% всего времени, затрачиваемого на машинное обучение, приходится именно на эту операцию. 
	Так же это один из немногих алгоритмов, который позволяет эффективно задействовать все вычислительные ресурсы современных процессоров и графических ускорителей. 
	Поэтому не удивительно, что многие алгоритмы стараются свести к матричному умножению — дополнительная расходы, связанные с подготовкой данных, как правило с лихвой окупаются общим ускорением алгоритмов. 
	Перемножения матриц также используются в прикладной физике, математике, математической статистике и многих других прикладных науках
	
\end{document}