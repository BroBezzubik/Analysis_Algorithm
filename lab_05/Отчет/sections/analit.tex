\documentclass[../main.tex]{subfiles}

\begin{document}
	
	В данном разделе указана поставленная цель. Описаны задачи необходимые для решения поставленной цели. Рассмотрены 2 подхода выполнения программ: конвейерный и последовательный.
	
\subsection{Постановка задачи}
	
	Цель - необходимо разработать 2 версии одной программы, которая обрабатывает строки 2 подходами: последовательным и конвейерным.
	
	Для выполнения данной цели необходимо выполнить следующие задачи:
	\begin{enumerate}[1)]
		\item изучить:
		\begin{enumerate}
			\item последовательный подход;
			\item конвейерный подход.
		\end{enumerate}
		\item выбрать программу для реализации;
		\item реализовать выбранную программу двумя подходами, указанными выше;
		\item произвести исследования:
		\begin{enumerate}
			\item замерить время работы от:
				\begin{enumerate}
					\item количество слов;
					\item длины слов
				\end{enumerate}
			\item сравнить;
			\item cделать выводы.
		\end{enumerate}
		\item подвести итоги.
	\end{enumerate}

\subsection{Программа}

	Наша программа будет работать и выполнять следующие этапы обработки строк:
	
	\begin{enumerate}[1)]
		\item генерировать указанное количество слов в зависимости от входных параметров;
		\item приводить их к верхнему регистру;
		\item заменять все буквы B на A.
	\end{enumerate}
	
\end{document}